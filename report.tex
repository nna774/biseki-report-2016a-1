\documentclass[12pt,a4]{article}

\setlength{\textwidth}{17cm}
\setlength{\textheight}{24cm}
\setlength{\leftmargin}{-1cm}
\setlength{\topmargin}{-2cm}
\setlength{\oddsidemargin}{0cm}
\setlength{\evensidemargin}{0cm}

\usepackage{xltxtra}
\setmainfont{IPAPMincho}
\setsansfont{IPAPGothic}
\setmonofont{IPAGothic}
\XeTeXlinebreaklocale "ja"

\usepackage{listings}
\usepackage{verbatim}
\usepackage{amsmath}

\title{微分積分学続論IIレポート課題1}
\date{}

\begin{document}
\maketitle

\fbox{1}

\begin{equation}
  A=
  \begin{pmatrix}
    1 & 0 \\
    1 & 2
  \end{pmatrix}
\end{equation}
の固有多項式
\begin{equation}
  \begin{vmatrix}
    tI-A
  \end{vmatrix}
  =
  \begin{vmatrix}
    t - 1 & 0 \\
    -1    & t - 2
  \end{vmatrix}
  = (t-1)(t-2)
\end{equation}
より、Aの固有値は1、2。

1に対する固有ベクトルは、
$
\left(
\begin{array}{c}
  x \\
  y
\end{array}
\right)
\neq
\left(
\begin{array}{c}
  0 \\
  0
\end{array}
\right)
$
なる
$
\left(
\begin{array}{c}
  x \\
  y
\end{array}
\right)
$ で、
\begin{equation}
  \begin{pmatrix}
    0 & 0 \\
    -1 & -1
  \end{pmatrix}
  \left(
  \begin{array}{c}
    x \\
    y
  \end{array}
  \right)
  = 0
\end{equation}
となるようなものなので、
$
\left(
\begin{array}{c}
  \frac{1}{\sqrt{2}}\\
  \frac{-1}{\sqrt{2}}
\end{array}
\right)
$などが取れる。

2に対する固有ベクトルは、
$
\left(
\begin{array}{c}
  x \\
  y
\end{array}
\right)
\neq
\left(
\begin{array}{c}
  0 \\
  0
\end{array}
\right)
$
なる
$
\left(
\begin{array}{c}
  x \\
  y
\end{array}
\right)
$ で、
\begin{equation}
  \begin{pmatrix}
    -1 & 0 \\
    -1 & 0
  \end{pmatrix}
  \left(
  \begin{array}{c}
    x \\
    y
  \end{array}
  \right)
  = 0
\end{equation}
となるようなものなので、
$
\left(
\begin{array}{c}
  0\\
  1
\end{array}
\right)
$
などが取れる。

$ P =
\begin{pmatrix}
  \frac{1}{\sqrt{2}} & 0 \\
  \frac{-1}{\sqrt{2}} & 1
\end{pmatrix}
$
とすると、
$ P^{-1} =
\begin{pmatrix}
  \sqrt{2} & 0 \\
  1 & 1
\end{pmatrix}
$ で、

$ P^{-1} A P =
\begin{pmatrix}
  1 & 0 \\
  0 & 2
\end{pmatrix}
$ となるので、
$
A^n = P \begin{pmatrix}
  1 & 0 \\
  0 & 2
\end{pmatrix}^n P^{-1}
=
\begin{pmatrix}
  1 & 0 \\
  2^n -1 & 2^n
\end{pmatrix}
$
\end{document}
